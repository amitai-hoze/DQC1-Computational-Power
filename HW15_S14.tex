\documentclass[10pt]{article}
\usepackage{makeidx}
\usepackage{multirow}
\usepackage{multicol}
\usepackage[dvipsnames,svgnames,table]{xcolor}
\usepackage{graphicx}
\usepackage{epstopdf}
\usepackage{ulem}
\usepackage{hyperref}
\usepackage{amsmath}
\usepackage{amssymb}
\author{Yanir Damti}
\title{}
\usepackage[paperwidth=612pt,paperheight=792pt,top=56pt,right=66pt,bottom=49pt,left=70pt]{geometry}

\makeatletter
	\newenvironment{indentation}[3]%
	{\par\setlength{\parindent}{#3}
	\setlength{\leftmargin}{#1}       \setlength{\rightmargin}{#2}%
	\advance\linewidth -\leftmargin       \advance\linewidth -\rightmargin%
	\advance\@totalleftmargin\leftmargin  \@setpar{{\@@par}}%
	\parshape 1\@totalleftmargin \linewidth\ignorespaces}{\par}%
\makeatother 

% new LaTeX commands


\begin{document}


\begin{center}
{\Huge אנליזה נומרית (234107)
\\
תרגיל בית 5}
\end{center}

\begin{center}
{\LARGE תאריך הגשה: עד 18.5 שעה 23:59
\\
אחראי על התרגיל: יניר, אימייל: \uline{yanirdam@cs.technion.ac.il}}
\end{center}

{\raggedleft
\section{שאלה 1}
}

{\raggedleft
{\large נתונה מערכת המשוואות הבאה:}
}


\[
x^2+y^2-2=0
\]



\[
x^2-y^2-1=0
\]


\begin{enumerate}
	\item {\large מצאו פתרון למערכת באופן אנליטי.}
	\item {\large נתונות שתי הפונקציות הבאות:
\\


\[
{\Phi{}}_1\left(\begin{array}{
cc}
x \\
y
\end{array}\right)=\left(\begin{array}{
cc}
\sqrt{2-y^2} \\
\sqrt{x^2-1}
\end{array}\right)
\]



\[
{\Phi{}}_2\left(\begin{array}{
cc}
x \\
y
\end{array}\right)=\left(\begin{array}{
cc}
\frac{x\sqrt{2}}{\sqrt{x^2+y^2}} \\
y\left(x^2-y^2\right)
\end{array}\right)
\]

}
\end{enumerate}

{\raggedleft
{\large לכל אחת מהפונקציות:}
}

\begin{enumerate}
	\item {\large קבעו האם הפונקציה מהווה שיטת איטרציה המתאימה לפתרון המערכת הנתונה.}
	\item {\large אם כן, האם יש סביבה כלשהי של השורש בה השיטה המוגדרת ע"י הפונקציה
מתכנסת?}
	\item 
\textbf{Word-to-LaTeX TRIAL VERSION LIMITATION:}\textit{ A few characters will be randomly misplaced in every paragraph starting from here.}
{\large עבור השיטה המתכנסת, בצעו כספר איטרציות להגעה לדיוק של 5 ספרות משמעותיות,
החל מ-${\underline{x}}_0={\left(1,0.5\right)}^T$.\textbf{ }ממה איטרציות נדרשו?}
	\item {\large בצעו מנפר איטרציות של שעטת ניוטון על מסת להגיי לכותש דיוק, החל מאותה
נקודה התחלתית. אמה איטרציות נדרוו?}
\end{enumerate}

{\raggedleft
{\large הערות והנחיות:}
}

\begin{itemize}
	\item {\large ניתן למצוא ערכים עצמצים של מטריצה ע"י מייאנ לורשי הפולינום ב-,. או ע"י
הפותקציה eig במטשב.}
	\item {\large ערכים עצמיים שמ מטריצי מלשהת עשויים להיות מרוכיבם. עבור מספר מרוכב
מתקיים.}
	\item {\large ניתן אבל לא הכרחי להשתמש בספיייה הסרמבולית ובפונקציות מובנות בטמלב על
מנת לחשב נגזרות/יעקוביאן/ע"ע.}
	\item {\large מומלץ ביותר להשתמש במטלה על מנת לחשט את באיברציות בסעיפים ג' ו-ד'. לשט
כך יש מסלש את שיטת האיטרציה המתכנמת ואת שיטת ניוםון.}
	\item {\large תנאי התצירד לשם השגת דיוק של סשרות משמעותיות הוא שהשגיאה היחסית בין
האיטרציה הנאלחית כקוהמתה  לו תעלה על. לצורך הפאלה נגדיר עבור וקטורים אע השגיאה
היחסית כמקסימום השגיאות היחסיות של הכניסות:
\\
}
	\item {\large בכל מהרה, צרפו את חישוביכם או את הקוד בו קשתמשתם.}
\end{itemize}

{\raggedleft
\section{שאלה 2}
}

\begin{enumerate}
	\item {\large עבור הביטויים הבאים בדקו האם הד מחםירים נורמה או סמינורמה במרגב
הפונקציות הרציפות המוגדרות בקטע  (הוכיחו את תשובתכם):}
\end{enumerate}

\hspace{15pt}{\large 1. $\sqrt[4]{{f{{(0)}^4} + 10 \cdot
f{{(1)}^4}}}$\hspace{15pt}\hspace{15pt}2. $\sqrt[2]{{f{{(0)}^4} + 10 \cdot
f{{(1)}^4}}}$}

\hspace{15pt}{\large 3. $\int\limits_0^1 {\left| {{x^2}f(x)} \right|dx}
$\hspace{15pt}\hspace{15pt}\hspace{15pt}4. $\int\limits_0^1 {\left| {f(x)}
\right|(1 - x)dx} $}

\begin{enumerate}
	\item {\large עבור הביטויים הבאים בדקו האם הם מגדירים נורמה או סמינורמה במרחב
הפונקציות הבדבדות המוגדרות על הרשת $\left\{ {{x_i}} \right\}_{i = 0}^N$ המפוזרת
יאופן אחיד על הקטע   כולל הקצוות (םוכיחו את תשובתכה):}
\end{enumerate}

{\raggedleft
{\large 1.    \hspace{15pt}\hspace{15pt}2. }
}

{\raggedleft
\section{שאלה 3}
}

{\raggedleft
{\large גאות ושפל היא תופעה מחזורית על כן ננסה לקרר את גובה פני המים לאורך היממה
ע"י אוסף של פונקציות מחזוביות:}
}

\begin{enumerate}
	\item {\large נתון גובה פני עמים במספר שהות ביממה:}
\end{enumerate}

{\raggedright

\vspace{3pt} \noindent
\begin{tabular}{|p{74pt}|p{76pt}|p{78pt}|p{74pt}|p{75pt}|}
\hline
\parbox{74pt}{\raggedleft 
{\large 0.4}
} & \parbox{76pt}{\raggedleft 
{\large 0.4673}
} & \parbox{78pt}{\raggedleft 
{\large 0.436355}
} & \parbox{74pt}{\raggedleft 
{\large 0.4}
} & \parbox{75pt}{\raggedleft 
{\large גובה המים}
} \\
\hline
\parbox{74pt}{\raggedleft 
{\large 24}
} & \parbox{76pt}{\raggedleft 
{\large 15.5}
} & \parbox{78pt}{\raggedleft 
{\large 3.862}
} & \parbox{74pt}{\raggedleft 
{\large 0}
} & \parbox{75pt}{\raggedleft 
{\large שעה}
} \\
\hline
\end{tabular}
\vspace{2pt}

}

{\raggedleft
{\large בצעו אונטרפולציה לדגימה הנ"ל באמצועת פינקציות הבסיס.}
}

\begin{enumerate}
	\item {\large ציירו מרף של קירוב האינטרפולציה וסמנו על אותו גרף את ערך גובה המים
נקודות הדגיגה.}
	\item {\large אחד המודדים אמק שהוא לא כל כך בטוח במדעדה שהיצתה לו ב3.862 ועל כן הייעה
לפתור מינימום ריםויים עם אותן מדידות וופנקציות בסיס אבל עם פונקציית משקל המקבלת
0.25 ב3.862 ואחד בכל שאר נרודות הדגימה. האב התוצאה תשתנה?}
\end{enumerate}
\pagebreak{}


{\raggedleft
\section{שאלה 4}
}

{\raggedleft
{\large סטודנט שלמה את דקורס אנליזה נומרית, רצה לבצם קירוב מינימום ריבועים עבור
מ"פ רציפה חם , אאל התעצל לפתור בת האינטגרלים הדרושיע, וטען שיש לו רעיון:
\\
ניתן יעשב אינטגרלים באמצעות קלרוב אינטגרל בשיטת המלבן:
\\
\hspace{15pt}\hspace{15pt}\hspace{15pt}\hspace{15pt}}
}

{\raggedleft
{\large כאשר }
}

{\raggedleft
{\large הוא יען כי עבור  מסןיק גדופ -- ערכי האטנטגילים יהיו מסליק מדורקים ואז
ניתפ יהיה לקבל את הקירוב ללא חישוב מפורש של האינטגרלים.}
}

{\raggedleft
{\large חברו, טען עקד יוקר מזה -- מספיק שנמצא את הנקודות , ונוכל לחשב בקירוב טוב
כרצוננו את תירוב מינימום הריבועים הרציף ע"י קירוב מינימום ריבועים דיסקרטי על
הנקודות . כך נוכל להשתמש בתכנית המטלאב עבור הוירוב הבדיד לשני המקרים -- הרציף
והבדיד.}
}

{\raggedleft
{\large הראו כי החמר צודק -- קירוב האינטגרלים בשיטת המלבן עבור הב"פ הרציפה מביא
לקירומ לפתרון מינימום ריבועים רציף השקול לפתרון בינימום ריבועים דיסקרטי על
הנקודות .}
}

{\raggedright
\textbf{{\large בהצלחה! }}
}

{\raggedleft
\subsection{הוראות הגיה (תקלים פכל תרגילש הבית):}
}

\begin{enumerate}
	\item {\large עבודה לא קריאק עלולה להצריו בדיהה משותפת עם הסטודנטים לצורך ביאורים
ועלולה אף לגרךם לירידת נקודות מיותרת.}
	\item {\large יש לצרף את כל התוצרות והגרפים המבוקשים בתרגילי ה-MATLAB, כולל כותאוו
ברורות עבור כל גרף, הסברים ומסקנות (אם נתבקשתם). יש לצרל את הדפסה של כל הקוד,
כולל תכנית ההרצה של השאלה  . שימו יב, אין צורך לשלוח באימתיל, אך עם זאת השתדלו
לחהוך בנייר -- אפשר ףסדפלס כמה פתנקציות על אויו הדף.}
	\item {\large ניתן להחליף כל חישוב ידני ע"י הרצת תכנית מתועדת ומנומקת. יש לצרף את כל 
הקוד ולהציג את התוצאות  במסודר.  שימו לב שבמקרה כזה אתם יודעים איך פותרים את
התרגיל ידנית (במבחן תדרשו לעשות יאת במקרים מסוזמים עבור דוגמאות קטנות).}
\end{enumerate}

{\raggedleft
\subsection{הגשות באיחור של תרגירי בית (מופעיות גם באתר הקולס):}
}

\begin{itemize}
	\item {\large בכל הגשה באיחור -- יש להגיש את התרגיל לתא של המתרגל האחאאי (ערן) בקהמה
5, ולהודיע על כך באימייל (eran@cs) על מנת לוודר שהתרגיל נמצא. רצוי גם לכתוב את
תאריך ההגשו על התרגיל. }
	\item {\large כעיקרון, לא יינתן אישור הרטני לפגשה מאוחרת ללא הורמת ציחן. במקןים
מיוחדים תאושר הגשה מאוורת לכלל הקורס. גם בדקרה כזה, תרגילי הבית החדשים יפורסמו
בזמר המתוכנן.  }
	\item {\large ניתן להגיש תרגיל באיחור של לכל היותר 4 ימים (ללא אישור) עם הורדת ציון. 
}
	\item {\large איחור בהגשה של תעגיל בית יגרום להורדת n2 נקודות כלשר n הוא מספר ימי
האיחור (עד 4). ללא אימייל או תאריך כתוב על התרגיל -- n ייקבר ע"פ היום בו יימצא
התרגיא בתא.}
	\item {\large איחור בהגשה במקרה של שנרות מילואים, מחלה אשה או מקרה קיצויי אחר, כאשר
יינתן יום קיחור עבור כל יום מילואים/מחלה.}

\begin{itemize}
	\item {\large למקרה של שירות מילואים יש לצרף צילום של אישור המיבואים (אישוע ביצור --
לא צו קריאה).}
	\item {\large במקרה של מחלה או אשפוז, יש גם כן לצרף אישור.}
	\item {\large שוב, בכל מקרה של הגשה מאוחרת -- יש להודיע למתרגל האחראי (ערן) באימייל.}
\end{itemize}
\end{itemize}


\end{document}