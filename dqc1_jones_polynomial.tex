% Created 2015-05-26 Tue 08:18
\documentclass[leqno,fleqn]{beamer}
\tolerance=1000
\input{qc_beamer}
\usepackage{amsmath}
\usepackage{beamerthemesplit} % new 
\usepackage{braket}
\usepackage{tikz}
\usepackage{braids}

\usetheme{PaloAlto}
\author{Amitai}
\date{\today}
\title{DQC1 complexity class and the trace estimation problem}
\hypersetup{
 pdfauthor={Amitai},
 pdftitle={DQC1 complexity class and the trace estimation problem},
 pdfkeywords={},
 pdfsubject={},
 pdfcreator={Emacs 24.3.1 (Org mode 8.3beta)}, 
 pdflang={English}}
\begin{document}

\maketitle
\begin{frame}{Outline}
\tableofcontents
\end{frame}

\section{The DQC1 complexity class}
\label{sec-1}
\begin{frame}[label=sec-1-1]{The DQC1 complexity class}
DQC1 class is the class of decidable languages with algorithm \(A\) such that:

\begin{itemize}
\item \(A\) starts with one clean qubit in state \(\ket{0}\), and \(n\) qubits in
the maximally mixed state
\item \(A\) may perform any unitary operation
\item \(A\) can only perform a measurement of the clean qubit at the end of
the algorithm
\item \(A\) has no access to a classical computer, so its not promised that
\(P \subset DQC1\)
\item \(A\) cannot be invoked many times in parallel
\item \(A\) runs in polynomial time
\item \(\forall x\), \(A\) decides if \(x \in L\) correctly with probability of
at least \(\frac{2}{3}\)
\end{itemize}
\end{frame}
\begin{frame}[label=sec-1-2]{The trace estimation problem is in DQC1}
\begin{block}{Trace estimation problem}
Given a quantom circuit, what is the trace of its unitary operation?
\end{block}
\begin{block}{Completeness}
A language L is said to be "complete" in the class DQC1, if:
\begin{itemize}
\item \(L \in DQC1\)
\item \(\forall L_{0} \in DQC1\) there is a reduction from \(L_{0}\) to \(L\), such that the reduction algorithm is in DQC1
\end{itemize}
\end{block}
\end{frame}
\begin{frame}[label=sec-1-3]{The Hadamard test}
\inlineQcircuit{
  \lstick{\ket{0}} & \gate {H} & \ctrl{1} & \gate {H} & \meter & \qw \\
  \lstick{\psi} & {/} \qw & \gate {U} & {/} \qw & \qw & \qw \\
}

We will show that this circuit indeed calculates the trace of U
\begin{block}{After the first hadamard gate}
\begin{align*}
   \Ket{+}\psi = \frac{1}{\sqrt{2}}\Ket{0}\Ket{\psi} + \frac{1}{\sqrt{2}}\Ket{1}\Ket{\psi}
\end{align*}
\end{block}
\begin{block}{After the C-U operation}
\begin{align*}
   \frac{1}{\sqrt{2}}\Ket{0}\Ket{\psi} + \frac{1}{\sqrt{2}}\Ket{1}U\Ket{\psi}
\end{align*}
\end{block}
\begin{block}{After the final hadamard operation}
\begin{align*}
   \frac{1}{2}\Ket{0}\Ket{\psi} + \frac{1}{2}\Ket{1}\Ket{\psi}\ +\frac{1}{2}\Ket{0}U\Ket{\psi}\ -  \frac{1}{2}\Ket{1}U\Ket{\psi} = \\
   \frac{\Ket{\psi} + U\Ket{\psi}}{2}\Ket{0} + \frac{\Ket{\psi} - U\Ket{\psi}}{2}\Ket{1}
\end{align*}
\end{block}
\end{frame}
\begin{frame}[label=sec-1-4]{The trace estimation problem is in DQC1}
The start state of any DQC1 problem is one clean qubit (state \(\ket{0}\)), and \(n\)-qubits in the maximally mixed state. That is, the start state is \(\rho = \ket{0}\bra{0} \otimes \frac{I}{2^n}\) We can use the Hadamard test, which gets as input this state, in order to accurately estimate a trace of unitary operation U.
\end{frame}
\section{hardness of trace estimation}
\label{sec-2}
\section{adding few more clean bits dont give extra power}
\label{sec-3}
\section{References}
\label{sec-4}
\end{document}