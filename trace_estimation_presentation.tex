%this is the DQC1 presentation

\documentclass{beamer}
\usepackage{etex}
\usepackage{amsmath}
\usepackage{beamerthemesplit} % new 
\usepackage{braket}
\usepackage{tikz}
\usepackage{braids}
\usepackage{qcircuit}
\begin{document}
\title{DQC1 complexity class and the trace estimation problem} 
\author{Ohad Barta} 
\date{\today} 

\frame{\titlepage} 
\begin{frame}[allowframebreaks]
\frametitle{Table of contents}
{\tableofcontents}
\end{frame}

\section{adding few more clean bits dont give extra power}
\frame{\frametitle{adding few more clean bits dont give extra power}
\begin{itemize}
\item We notice now that actually we didn't computed the trace accurately. Since only the expectation of the algorithm is the trace, we rather get an approximation. 
\item We measure a polynimial number of points on the trace. Therefore, the variance of our estimation is O($\frac{1}{poly(n)}$) , and therefore with high probability we get a  O( $\frac{1}{poly(n)}$) estimation.
\end{itemize}
}

\frame{\frametitle{adding few more clean bits dont give extra power(2) }
\begin{itemize}
\item On the other hand, an approximation of $\frac{1}{poly(n)}$ to the expression $\frac{Tr(U)}{2^{n+1}}$ is enough to decide every problem in DQC1 using the analysis above
\item Therefore, we showed that getting a  $\frac{2^{n}}{poly(n)}$ additive- approximation to the trace is a DQC1-complete problem
 
\end{itemize}
. 
}

\frame{\frametitle{adding few more clean bits dont give extra power (3)}
We will define now the DQCK complexity class:
DQCK is the same as DQC1, except the following differences:
\begin{itemize}
\item A's start state include K clean qubits. In case that the input $x \in L$, we will measure 0 in the first clean qubit at the end of A with probability of at least $\frac{2}{3}$
\end{itemize}  
}

\frame{\frametitle{adding few more clean bits dont give extra power (4)}
We will now prove that for $k \leq \log{n}$, estimate the trace of unitary matrix with the same precision is still a complete problem \cite{SJ2008}, thus proving that adding logarithmic number of clean bits doesn't change the power. 

Obviously we can calculate the trace of unitary matirx with $\log{n}$ bits, since we can do it just with one. 
}

\frame{\frametitle{adding few more clean bits dont give extra power (3)}
As for the less trivial direction, assume we have some quantum algorithm in DQCK.
Similarly to the one-qubit option, final state is:
$\rho_{final} = U\rho{U^\dagger} = U{\ket{0}\bra{0}}^{\otimes k}\frac{I}{2^n}{U^\dagger}$
and therefore the probability of measuring 0 at the end is:
$p_{0} = Tr[{\ket{0}\bra{0}} \otimes I\rho_{final}] = 2^{-n}Tr[(\ket{0}\bra{0}\otimes I)U({\ket{0}\bra{0}}^{\otimes k}\otimes I{U^\dagger})$

Now, we have the same problem at estimatimg this matrix: its not unitary!
To resolve that issue, we build circuit similar to the one in the 1-clean quibit process,
but now we add additional  $k-1$ ancila qubits, when there is a CNOT gate between the 2...k clean qubits, and the corresponding ancilla qubits.
(Thus, enforcing them to be zero in order to contribute to the circuit's trace)/
}

\frame{\frametitle{adding few more clean bits dont give extra power (5)}
\Qcircuit @C=1em @R=.7em {
	& {/}  \qw &  \qw & \multigate{1}{U^\dag} & \ctrl{2} & \multigate{1}{U} & \ctrl{3} & \qw & \qw\\
	& {/}  \qw & \qw & \ghost{U^\dag} & \qw & \ghost{U} & \qw  & \qw   {/} & \qw \\
	&  {/}   \qw & \qw &\qw &  \targ  & \qw & \qw & \qw  &  {/} \qw \\
	&  \qw & \qw & \qw & \qw & \qw & \targ & \qw  & \qw
}
Now, we can see (similary to the proposition 1), that the trace of the new circuit $U^{*}$ follows the rule: $Tr[U^{*}] = 2^{k}Tr[U]$. Thus, in polynomial number of executions we can compute its trace up to a percision of $\frac{2^{n+k}}{poly(n,k)}$, but this equals to $\frac{2^{n}}{poly(n)}$ when $k  \leq \log{n}$, which means that in this case the precision is good enough to decide the original problem.
}
\section{References}
\frame{\frametitle{References}
\begin{thebibliography}{9}

\bibitem{SJ2008}
   Peter W. Shor,
   Stephen P. Jordan,
  Estimating Jones Polynomials is a Complete Problem for One Clean Qubit,
   2008.
\end{thebibliography}
}
\end{document}